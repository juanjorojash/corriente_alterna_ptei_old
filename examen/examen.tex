\documentclass[12pt]{exam}
\usepackage[utf8]{inputenc}
\usepackage[spanish,activeacute]{babel}
\usepackage[margin=1in]{geometry}
\usepackage{amsmath,amssymb}
\usepackage{multicol}
\usepackage{siunitx}
\usepackage[american]{circuitikz} %Este se usa para hacer circuitos
\usepackage{float}
\usetikzlibrary{babel}
\newcommand{\class}{Técnico en Electricidad Industrial}
\newcommand{\term}{Electricidad I}
\newcommand{\examnum}{Examen Parcial II}
\newcommand{\examdate}{15/10/2019}
\newcommand{\timelimit}{180 Minutos}


% *** GRAPHICS RELATED PACKAGES ***
%
\usepackage{graphicx}
\graphicspath{{../pdf/}{../png/}}
\DeclareGraphicsExtensions{.pdf,.jpg,.png}
\usepackage{subfigure}
\pagestyle{head}
\firstpageheader{}{}{}
\runningheader{\examnum}{Pagina \thepage\ de \numpages}{\examdate}
\runningheadrule

\pointpoints{punto}{puntos}
\begin{document}
% Cambio de nombre de cuadro a Tabla
\renewcommand{\listtablename}{Índice de tablas}
\renewcommand{\tablename}{Tabla}

\noindent
\begin{tabular*}{\textwidth}{l @{\extracolsep{\fill}} r@{\extracolsep{6pt}}}
\textbf{Nombre:} \makebox[3in]{\hrulefill}&\\	
& \textbf{\examnum}\\
\textbf{\class}& \textbf{\examdate}\\
\textbf{Fundatec}& \textbf{\term} \\
\textbf{Tecnológico de Costa Rica}& \textbf{Tiempo: \timelimit} \\
\end{tabular*}\\
\rule[2ex]{\textwidth}{2pt}
\textbf{Instrucciones}
\begin{itemize} \itemsep1pt \parskip0pt \parsep0pt
\item Documente adecuadamente los pasos necesarios para llegar a la respuesta. 
\item Este examen consta de \numpages\ paginas (incluyendo esta) y \numquestions\ preguntas.
El numero total de puntos es \numpoints.
\end{itemize}



%\begin{center}
%Tabla de calificación (No rayar)\\
%\addpoints
%\gradetable[v][questions]
%\end{center}

\noindent
\rule[2ex]{\textwidth}{2pt}


\addpoints
\begin{questions} 

\question[2] Para el siguiente circuito indique el valor de la resistencia total del circuito entre los puntos \emph{a} y \emph{b}.
\begin{center}
	\begin{circuitikz}
		\draw
		(0,0) to[R, l=\SI{300}{\ohm}] (0,3)
		to[short] (2,3)
		to[R, l=\SI{200}{\ohm}] (2,0)
		to[short] (0,0)
		;
		\draw 
		(2,3) to[short, -o] (3.5,3) node[right]{a} 
		;	
		\draw 
		(2,0) to[short, -o] (3.5,0) node[right]{b} 
		;	
	\end{circuitikz}
\end{center}	

\addpoints	
\question[2] Para el siguiente circuito indique el valor de la resistencia total del circuito entre los puntos \emph{a} y \emph{b}.
\begin{center}
\begin{circuitikz}
	\draw
	(0,0) to[R, l=\SI{235}{\ohm}] (0,3)
	to[R, l=\SI{115}{\ohm}] (3,3)
	to[short, -o] (3.5,3) node[right]{a}
	;
	\draw 
	(0,0) to[short, -o] (3.5,0) node[right]{b} 
	;	
\end{circuitikz}
\end{center}

\addpoints
\question[1] Suponga que usted arma el circuito de la pregunta 2 y coloca el multímetro en la función de medición de resistencia con la punta común (COM) en \emph{a} y la otra punta en \emph{b}. Luego de esto usted obtiene un valor de \SI{341}{\ohm}. Calcule el porcentaje de error de la medición realizada y llene la siguiente tabla.
\vspace{0.5cm}
\begin{center}
\begin{tabular}{|c|c|c|}
	\hline
	Teórico & Experimental & \% de Error \\
	\hline
	& \SI{341}{\ohm}& \\
	\hline
\end{tabular}
\end{center}


%8.86
\addpoints
\question[4] Encuentre las corrientes de malla para el siguiente circuito.
\question[1] Determine el valor de la corriente $I_x$.
\question[1] Determine el valor del voltaje $V_x$.
\begin{center}
	\begin{circuitikz}
		\draw
		(0,0) 
		to[V=\SI{10}{\volt}, invert] (0,4)
		to[R=\SI{5}{\kilo\ohm}] (4,4)
		to[R=\SI{3}{\kilo\ohm}, i=$I_x$] (4,0)
		to[short] (0,0)
		;
		\draw 
		(4,4) 
		to[R=\SI{4}{\kilo\ohm}] (8,4)
		to[R=\SI{2}{\kilo\ohm}, v=$V_x$] (8,0)
		to[V=\SI{5}{\volt}] (4,0)
		;		
	\end{circuitikz}
\end{center}

\end{questions}

\end{document}
