\documentclass[12pt]{article}
\usepackage[margin=1in]{geometry} 
\usepackage{amsmath}
\usepackage{tcolorbox}
\usepackage{amssymb}
\usepackage{amsthm}
\usepackage{lastpage}
\usepackage{fancyhdr}
\usepackage{accents}
\usepackage{siunitx}
\usepackage[american]{circuitikz}
\pagestyle{fancy}
\setlength{\headheight}{40pt}

\begin{document}

\lhead{Técnico en Electricidad Industrial \\ Fundatec \\ Tecnológico de Costa Rica} 
\rhead{Corriente Alterna - 2019 \\ Tarea #2  \\ Entrega: 30/11/2019} 
\cfoot{\thepage\ de \pageref{LastPage}}
\noindent\textbf{Ejercicio único}. Para el circuito mostrado, si se tiene que:  $f=\SI{60}{\hertz}$, $L=\SI{2}{\milli\henry}$, $R = \SI{1}{\ohm}, $ $I_T=2\,\angle\,0 \si{\degree}$\,A. Calcule las siguientes incógnitas y de su respuesta con magnitud y ángulo:
\begin{itemize}
    \item[] $X_L$, $V_L$ y $V_R$
\end{itemize}
Utilizando los datos anteriores, realice las gráficas fasoriales vistas en clase pero a escala, de forma que pueda realizar medidas de longitudes y ángulos para encontrar las siguientes incógnitas:
\begin{itemize}
    \item[] $V_T$ y $Z_T$
\end{itemize}

\begin{center}
	\begin{circuitikz}
		\draw
		(0,0) to[sV, l=$V_T$] (0,3)
		to[R, l=$R$, i>^=$I_T$] (4,3)
		to[L, l=$L$] (4,0)
		to[short] (0,0)
		;
	\end{circuitikz}
\end{center}	

\noindent\textbf{Aclaraciones:}
\begin{itemize}
    \item Se deben realizar dos gráficas, una para calcular $V_T$ y la otra para calcular $Z_T$. 
    \item Cualquier entrega tardía se califica por debajo de 70. 
\end{itemize}
\end{document}