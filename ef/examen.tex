\documentclass[12pt]{exam}
\usepackage[utf8]{inputenc}
\usepackage[spanish,activeacute]{babel}
\usepackage[margin=1in]{geometry}
\usepackage{amsmath,amssymb}
\usepackage{multicol}
\usepackage{siunitx}
\usepackage[american]{circuitikz} %Este se usa para hacer circuitos
\usepackage{float}
\usetikzlibrary{babel}
\newcommand{\class}{Técnico en Electricidad Industrial}
\newcommand{\term}{Corriente Alterna}
\newcommand{\examnum}{Examen Final}
\newcommand{\examdate}{14/12/2019}
\newcommand{\timelimit}{180 Minutos}


% *** GRAPHICS RELATED PACKAGES ***
%
\usepackage{graphicx}
\graphicspath{{../pdf/}{../png/}}
\DeclareGraphicsExtensions{.pdf,.jpg,.png}
\usepackage{subfigure}
\pagestyle{head}
\firstpageheader{}{}{}
\runningheader{\examnum}{Pagina \thepage\ de \numpages}{\examdate}
\runningheadrule

\pointpoints{punto}{puntos}
\begin{document}
% Cambio de nombre de cuadro a Tabla
\renewcommand{\listtablename}{Índice de tablas}
\renewcommand{\tablename}{Tabla}

\noindent
\begin{tabular*}{\textwidth}{l @{\extracolsep{\fill}} r@{\extracolsep{6pt}}}
\textbf{Nombre:} \makebox[3in]{\hrulefill}&\\	
& \textbf{\examnum}\\
\textbf{\class}& \textbf{\examdate}\\
\textbf{Fundatec}& \textbf{\term} \\
\textbf{Tecnológico de Costa Rica}& \textbf{Tiempo: \timelimit} \\
\end{tabular*}\\
\rule[2ex]{\textwidth}{2pt}
\textbf{Instrucciones}
\begin{itemize} \itemsep1pt \parskip0pt \parsep0pt
\item Documente adecuadamente los pasos necesarios para llegar a la respuesta. 
\item Este examen consta de \numpages\ paginas (incluyendo esta) y \numquestions\ preguntas.
El numero total de puntos es \numpoints.
\end{itemize}



%\begin{center}
%Tabla de calificación (No rayar)\\
%\addpoints
%\gradetable[v][questions]
%\end{center}

\noindent
\rule[2ex]{\textwidth}{2pt}


\addpoints
\begin{questions} 

\question[2] Para el siguiente circuito indique el valor de la inductancia total del circuito entre los puntos \emph{a} y \emph{b}.
\begin{center}
	\begin{circuitikz}
		\draw
		(0,0) to[L, l=\SI{10}{\milli\henry}] (0,3)
		to[short] (2,3)
		to[L, l=\SI{20}{\milli\henry}] (2,0)
		to[short] (0,0)
		;
		\draw 
		(2,3) to[short, -o] (3.5,3) node[right]{a} 
		;	
		\draw 
		(2,0) to[short, -o] (3.5,0) node[right]{b} 
		;	
	\end{circuitikz}
\end{center}	

\addpoints	
\question[2] Para el siguiente circuito indique el valor de la capacitancia total del circuito entre los puntos \emph{a} y \emph{b}.
\begin{center}
\begin{circuitikz}
	\draw
	(0,0) to[C, l=\SI{50}{\micro\farad}] (0,3)
	to[C, l=\SI{100}{\micro\farad}] (3,3)
	to[short, -o] (3.5,3) node[right]{a}
	;
	\draw 
	(0,0) to[short, -o] (3.5,0) node[right]{b} 
	;	
\end{circuitikz}
\end{center}

\addpoints
\question[1] Suponga que usted arma el circuito de la pregunta 2 y coloca el multímetro en la función de medición de capacitancia con la punta común (COM) en \emph{b} y la otra punta en \emph{a}. Luego de esto usted obtiene un valor de \SI{153}{\micro\farad}. Calcule el porcentaje de error de la medición realizada y llene la siguiente tabla.
\vspace{0.5cm}
\begin{center}
\begin{tabular}{|c|c|c|}
	\hline
	Teórico & Experimental & \% de Error \\
	\hline
	& \SI{153}{\micro\farad}& \\
	\hline
\end{tabular}
\end{center}


%8.86
\addpoints
\question[5] Para el circuito mostrado abajo, si se tiene que:  $f=\SI{60}{\hertz}$, $R = \SI{1}{\ohm}$, $L=\SI{2}{\milli\henry}$, $C = \SI{5}{\milli\farad}$, $I_T=2\,\angle\,0 \si{\degree}$\,A. Calcule las siguientes incógnitas y de su respuesta con magnitud y ángulo:
\begin{itemize}
    \item[] $X_L$, $X_C$, $V_R$, $V_L$ y $V_C$
\end{itemize}
\question[6] Utilizando los datos anteriores, realice las gráficas fasoriales vistas en clase y encuentre las siguientes incógnitas:
\begin{itemize}
    \item[] $V_T$, $Z_T$, P, Q, S y fp
\end{itemize}
\question[2] Indique que valor debe tener un capacitor (colocado en paralelo con $V_T$)  para lograr un factor de potencia unitario (fp = 1)

\begin{center}
	\begin{circuitikz}
		\draw
		(0,0) to[sV, l=$V_T$] (0,3)
		to[R, l=$R$, i>^=$I_T$] (4,3)
		to[L, l=$L$] (4,0)
		to[C, l=$C$] (0,0)
		;
	\end{circuitikz}
\end{center}	

\end{questions}

\end{document}
