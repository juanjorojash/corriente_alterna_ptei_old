\documentclass[aspectratio=169]{beamer}
\usetheme{Bruno}
\usepackage{amsmath}
\usepackage{siunitx}
\usepackage[american,RPvoltages]{circuitikz}
\ctikzset{capacitors/scale=0.7}
\usepackage{tabularx}
\newcolumntype{C}{>{\centering\arraybackslash}X}
\usepackage{tabu}
\usepackage[spanish]{babel}
\usepackage{booktabs}
\usepackage{pgfplots}
\usepgfplotslibrary{units, fillbetween} 
\pgfplotsset{compat=1.16}
\renewcommand\tabularxcolumn[1]{m{#1}}% for vertical centering text in X column
\usetikzlibrary{arrows, arrows.meta}
\ctikzset{capacitors/scale=0.7}
\title{Electricidad I: \\ \emph{Leyes básicas}}
\author{
    Juan J. Rojas
}
\institute{Instituto Tecnológico de Costa Rica}
\date{\today}
\background{fig/background.jpg}
\begin{document}
\sisetup{unit-math-rm=\mathrm,math-rm=\mathrm} % change sinitx font
\sisetup{output-decimal-marker = {,}}
\maketitle

\begin{frame}{Corriente, Voltaje y Resistencia}
    \begin{center}
        \begin{tabularx}{12cm}{C C C}
        \toprule
        Variable & Derivación & Unidades \\
        \midrule
        Corriente & $i = \frac{dq}{dt}$ & \si{\ampere} [\si{\coulomb \second^{-1}}] \\[5pt]
        Voltaje & $v = \frac{dw}{dq}$ & \si{\volt} [\si{\joule \coulomb^{-1}}] \\[5pt]
        Potencia & $p = v \cdot i =  \frac{dw}{dt}$ & \si{\watt} [\si{\joule \second^{-1}}] \\[5pt]
        Energía & $w = \int_{t_0}^{t_f}p\,dt$ & \si{\watt\hour} [\SI{3600}{\joule}] \\[5pt]
        \bottomrule
        \end{tabularx}    
    \end{center}
\end{frame}

\begin{frame}{Ley de Ohm y Ley de Watt}
Ley de Ohm
    \begin{align*}
         V&=I \cdot R & I&=\frac{V}{R} & R&=\frac{V}{I}\\    
    \end{align*} 
Ley de Watt
    \begin{align*}
         P&=V \cdot I & P&=\frac{V^2}{R} & P&=I^2 \cdot R\\    
    \end{align*} 
\end{frame}

\begin{frame}{Pasivos y activos. Convención pasiva}
    \begin{tabularx}{\linewidth}{X X}
        \centering
        \begin{circuitikz} [scale=1]\draw
            (1,0)
                to[short,o-]
            (0,0)	
                to[generic]
            (0,3)
                to[short,-o,i<=$i$]
            (1,3)
                to[open,v=$v$]
            (1,0)
            ;
        \draw (0.5,4)node[above] {Pasivos};
        \draw (0.5,-1)node[above] {$p=v\cdot i$};
        \end{circuitikz}
        &
        \centering
        \begin{circuitikz} [scale=1]\draw
            (1,0)
                to[short,o-]
            (0,0)	
                to[generic]
            (0,3)
                to[short,-o,i=$i$]
            (1,3)
                to[open,v=$v$]
            (1,0)
            ;
        \draw (0.5,4)node[above] {Activos};
        \draw (0.5,-1)node[above] {$p=-v\cdot i$};
        \end{circuitikz}
    \end{tabularx}
\end{frame}

\begin{frame}{Fuentes independientes}
    \begin{tabularx}{\linewidth}{X X}
        \centering
        \begin{circuitikz} [scale=1]\draw
            (1,0)
                to[short,o-]
            (0,0)	
                to[V]
            (0,3)
                to[short,-o]
            (1,3)
            ;
        \draw (0.5,3.5)node[above, align=center]{Fuente \\de voltaje \\independiente};
        \end{circuitikz}
        &
        \centering
        \begin{circuitikz} [scale=1]\draw
            (1,0)
                to[short,o-]
            (0,0)	
                to[I]
            (0,3)
                to[short,-o]
            (1,3)
            ;
        \draw (0.5,3.5)node[above, align=center]{Fuente \\ de corriente \\independiente};
        \end{circuitikz}
    \end{tabularx}
\end{frame}

\begin{frame}{Fuentes dependientes}
    \begin{tabularx}{\linewidth}{X X}
        \centering
        \begin{circuitikz} [scale=1]\draw
            (1,0)
                to[short,o-]
            (0,0)	
                to[cV]
            (0,3)
                to[short,-o]
            (1,3)
            ;
        \draw (0.5,3.5)node[above, align=center]{Fuente \\de voltaje \\dependiente};
        \end{circuitikz}
        &
        \centering
        \begin{circuitikz} [scale=1]\draw
            (1,0)
                to[short,o-]
            (0,0)	
                to[cI]
            (0,3)
                to[short,-o]
            (1,3)
            ;
        \draw (0.5,3.5)node[above, align=center]{Fuente \\ de corriente \\dependiente};
        \end{circuitikz}
    \end{tabularx}
\end{frame}

\begin{frame}{Elementos de un circuito}
    \begin{tabularx}{\linewidth}{X X}
        \begin{itemize}
            \item \textbf{Rama:} un solo elemento
            \item \textbf{Nodo:} el punto de conexión entre dos o mas ramas
            \item \textbf{Lazo:} cualquier trayectoria cerrada en un circuito
            \item \textbf{Malla:} una lazo que no contiene otros lazos 
        \end{itemize}
        &
        \centering
        \begin{circuitikz} [scale=1]\draw
            (0,0)
                to[V, l=$v_1$]
            (0,3)	
                --
            (2,3)
                to[R,l=$R_2$]
            (4,3)
            (4,0) 
                to[V, l=$v_2$]
            (4,3)
            (0,0)
                --
            (4,0)
            (2,0)
                to[R, l=$R_1$]
            (2,3)
            ;
        \end{circuitikz}
    \end{tabularx}
%\let\thefootnote\relax\footnote{Tomado de \cite{charles2013fundamentos}}
\end{frame}


\begin{frame}{Ley de corriente de Kirchhoff (LCK)}
\emph{La suma algebraica de las corrientes que entran y salen de un nodo es igual a cero}
\vfill
\centering
        \begin{circuitikz} [scale=0.8,transform shape]\draw
            (0,0)
                to[generic,n=gen1]
            (0,2.5) -- (0,5) -- (3,5)
                to[generic,n=gen2]
            (6,5) -- (9,5) -- (9,2.5)
                to[generic,n=gen3]
            (9,0) -- (0,0)
            (0,2.5)
                to[generic,n=gen4]
            (3,2.5)
                to[generic,n=gen5]
            (6,2.5)
                to[generic,n=gen6]
            (9,2.5)
            (3,2.5)
                to[generic,n=gen7]
            (3,0)
            (6,2.5)
                to[generic,n=gen8]
            (6,0)
            (gen1.north)node[above,xshift=-0.2cm,rotate=90]{$i_1$}
            (gen2.north)node[above,yshift=+0.2cm]{$\SI{2}{\ampere}$}
            (gen3.north)node[above,xshift=0.2cm,rotate=-90]{$\SI{4}{\ampere}$}
            (gen4.north)node[above,yshift=+0.2cm]{$i_2$}
            (gen5.north)node[above,yshift=+0.2cm]{$\SI{7}{\ampere}$}
            (gen6.north)node[above,yshift=+0.2cm]{$i_4$}
            (gen7.north)node[above,xshift=0.2cm,rotate=-90]{$\SI{3}{\ampere}$}
            (gen8.north)node[above,xshift=0.2cm,rotate=-90]{$i_3$}
            ;
            \draw[-latex,thick] ($(gen1.north) + (-0.2,-0.5)$) -- ($(gen1.north) + (-0.2,0.5)$);
            \draw[-latex,thick] ($(gen2.north) + (-0.5,0.2)$) -- ($(gen2.north) + (0.5,0.2)$);
            \draw[-latex,thick] ($(gen3.north) + (0.2,0.5)$) -- ($(gen3.north) + (0.2,-0.5)$);
            \draw[-latex,thick] ($(gen4.north) + (0.5,0.2)$) -- ($(gen4.north) + (-0.5,0.2)$);
            \draw[-latex,thick] ($(gen5.north) + (-0.5,0.2)$) -- ($(gen5.north) + (0.5,0.2)$);
            \draw[-latex,thick] ($(gen6.north) + (0.5,0.2)$) -- ($(gen6.north) + (-0.5,0.2)$);
            \draw[-latex,thick] ($(gen7.north) + (0.2,0.5)$) -- ($(gen7.north) + (0.2,-0.5)$);
            \draw[-latex,thick] ($(gen8.north) + (0.2,0.5)$) -- ($(gen8.north) + (0.2,-0.5)$);
        \end{circuitikz}\cite{charles2013fundamentos}
\end{frame}

% \begin{frame}{Ley de corriente de Kirchhoff (LCK)}
% \emph{La suma algebraica de las corrientes que entran y salen de un nodo es igual a cero}\\
% \centering
% \includegraphics[width=0.8\linewidth]{fig/P2.13.PNG}\cite{charles2013fundamentos}
% \end{frame}

\begin{frame}{Ley de voltaje de Kirchhoff (LVK)}
\emph{La suma algebraica de todas los voltajes alrededor de un lazo es cero}
\vfill
\centering
        \begin{circuitikz} [scale=0.8,transform shape]\draw
            (0,0)
                to[generic, v^=$\SI{4}{\volt}$]
            (0,2.5) 
                to[generic, v^>=$\SI{3}{\volt}$]
            (0,5) -- (8,5)
                to[generic, v^>=$v_2$]
            (8,2.5)
                to[generic, v^=$\SI{5}{\volt}$]
            (8,0) -- (0,0)
            (0,2.5)--(0.75,2.5)
                to[generic, v=$v_3$]
            (3.25,2.5)--(4.75,2.5)
                to[generic, v=$\SI{2}{\volt}$]
            (7.25,2.5)--(8,2.5)
            (4,5)
                to[generic, v=$v_1$]
            (4,2.5)
                to[generic, v=$v_4$]
            (4,0)
            ;
        \end{circuitikz}\cite{charles2013fundamentos}
\end{frame}


% \begin{frame}{Ley de voltaje de Kirchhoff (LVK)}
% \emph{La suma algebraica de todas los voltajes alrededor de un lazo es cero}\\
% \centering
% \includegraphics[width=0.8\linewidth]{fig/P2.14.PNG}\cite{charles2013fundamentos}
% \end{frame}

\begin{frame}{Elementos en serie}
    \begin{tabularx}{\linewidth}{X X}
        \begin{itemize}
            \item Dos elementos en serie comparten solamente una de sus terminales y esta no se comparte con un tercer elemento.
            \item Cuando solo existen parejas de elementos en serie en un circuito se le llama circuito en serie.
            \item La corriente en un circuito en serie es la misma para todos los elementos.
        \end{itemize}
        &
        \centering
        \begin{circuitikz} [scale=1]\draw
            (3,0)
                to[R,l=$R_3$,o-]
            (0,0)	
                to[R,l=$R_2$]
            (0,3)
                to[R,l=$R_1$,i^<=$i$,-o]
            (3,3)
            ;
        \draw (1.5,-1.5)node[above, align=left]{$R_{eq}=R_1+R_2+R_3$};
        \end{circuitikz}
    \end{tabularx}
\end{frame}

\begin{frame}{Divisor de voltaje}
    \begin{tabularx}{\linewidth}{X X}
        \begin{gather*}
        v_n = v_T\cdot \frac{R_n}{R_1+R_2+\cdot\cdot\cdot+R_n}\\
        \\
        v_3 = v_T\cdot \frac{R_3}{R_1+R_2+R_3}
        \end{gather*}
        &
        \centering
        \begin{circuitikz} [scale=1]\draw
            (3,0)
                to[R,l=$R_3$,v>=$v_3$,o-]
            (0,0)	
                to[R,l=$R_2$]
            (0,3)
                to[R,l=$R_1$,-o]
            (3,3)
            (3,3)
                to[open,v^=$v_T$]
            (3,0)
            ;
        \end{circuitikz}
    \end{tabularx}
\end{frame}

\begin{frame}{Fuentes de voltaje en serie}
    \begin{tabularx}{\linewidth}{X X}
        \begin{circuitikz} [scale=1]\draw
            (3,0)
                to[V,l=$v_3$,o-]
            (0,0)	
                to[V,l=$v_2$, invert]
            (0,3)
                to[V,l=$v_1$,-o]
            (3,3)
            ;
        \end{circuitikz}
        &
        \centering
        \begin{circuitikz} [scale=1]\draw
            (1,0)
                to[short,o-]
            (0,0)	
                to[V,l=$v_e$]
            (0,3)
                to[short,-o]
            (1,3)
            ;
        \end{circuitikz}
    \end{tabularx}
    \begin{equation*}
    v_{eq} = v_1-v_2+v_3
    \end{equation*}
\end{frame}

\begin{frame}{Elementos en paralelo}
    \begin{tabularx}{\linewidth}{X X}
        \begin{itemize}
            \item Dos o mas elementos en paralelo comparten sus dos terminales.
            \item Cuando solo existen elementos en paralelo en un circuito se le llama circuito en paralelo.
            \item El voltaje en un circuito en paralelo es el mismo para todos los elementos.
        \end{itemize}
        &
        \centering
        \begin{circuitikz} [scale=1]\draw
            (1,3)
                to[short,o-]
            (0,3)	
                to[R,l_=$R_3$]
            (0,0)
                to[short,-o]
            (1,0)
            (0,3) -- (-1.5,3)
                to[R,l_=$R_2$]
            (-1.5,0) -- (0,0)
            (-1.5,3) -- (-3,3)
                to[R,l_=$R_1$]
            (-3,0) -- (-1.5,0)
            (1,3)
                to[open,v^=$v$]
            (1,0)
            ;
            \draw (-1.5,-3)node[above, align=center]{$\frac{1}{R_{eq}}=\frac{1}{R_1}+\frac{1}{R_2}+\frac{1}{R_3}$\\ \\En caso de solo dos resistencias:\\ \\ $R_{eq}=\frac{R_1 \cdot R_2}{R_1+R_2}$ };
        \end{circuitikz}
    \end{tabularx}
\end{frame}

\begin{frame}{Divisor de corriente}
    \begin{tabularx}{\linewidth}{X X}
        \begin{gather*}
        i_n = i_T\cdot \frac{\frac{1}{R_n}}{\frac{1}{R_1}+\frac{1}{R_2}+\cdot\cdot\cdot+\frac{1}{R_n}}\\
        \\
        i_1 = i_T\cdot \frac{\frac{1}{R_1}}{\frac{1}{R_1}+\frac{1}{R_2}+\frac{1}{R_3}}
        \end{gather*}
        En caso de ser solo dos resistencias:
        \begin{equation*}
        i_1 = i_T\cdot \frac{R_2}{R_1+R_2}
        \end{equation*}
        &
        \centering
        \begin{circuitikz} [scale=1]\draw
            (1,3)
                to[short,i=$i_T$,o-]
            (0,3)	
                to[R,l_=$R_3$]
            (0,0)
                to[short,-o]
            (1,0)
            (0,3) -- (-1.5,3)
                to[R,l_=$R_2$]
            (-1.5,0) -- (0,0)
            (-1.5,3) -- (-3,3)
                to[R,l_=$R_1$,i>^=$i_1$]
            (-3,0) -- (-1.5,0)
            ;
        \end{circuitikz}
    \end{tabularx}
\end{frame}

\begin{frame}{Fuentes de corriente en paralelo}
    \begin{tabularx}{\linewidth}{X X}
        \begin{circuitikz} [scale=1]\draw
            (1,3)
                to[short,o-]
            (0,3)	
                to[I,l_=$i_3$]
            (0,0)
                to[short,-o]
            (1,0)
            (0,3) -- (-1.5,3)
                to[I,l_=$i_2$, invert]
            (-1.5,0) -- (0,0)
            (-1.5,3) -- (-3,3)
                to[I,l_=$i_1$]
            (-3,0) -- (-1.5,0)
            ;
        \end{circuitikz}
        &
        \centering
        \begin{circuitikz} [scale=1]\draw
            (1,3)
                to[short,o-]
            (0,3)	
                to[I,l_=$i_e$,invert]
            (0,0)
                to[short,-o]
            (1,0)
            ;
        \end{circuitikz}
    \end{tabularx}
    \begin{equation*}
    i_{eq} = -i_1+i_2-i_3
    \end{equation*}
\end{frame}

\begin{frame}{Referencias}

\bibliographystyle{ieeetr}

\bibliography{referencias}

\end{frame}

\end{document}
