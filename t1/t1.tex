\documentclass[12pt]{article}
\usepackage[margin=1in]{geometry} 
\usepackage{amsmath}
\usepackage{tcolorbox}
\usepackage{amssymb}
\usepackage{amsthm}
\usepackage{lastpage}
\usepackage{fancyhdr}
\usepackage{accents}
\usepackage{siunitx}
\pagestyle{fancy}
\setlength{\headheight}{42pt}

\begin{document}

\lhead{Técnico en Electricidad Industrial \\ Fundatec \\ Tecnológico de Costa Rica} 
\rhead{Corriente Alterna \\ Tarea \#1  \\ Entrega: 14/03/2020} 
\cfoot{\thepage\ de \pageref{LastPage}}
\noindent\textbf{Ejercicio único}. Si se tiene que:  $f=\SI{80}{\hertz}$, $A_p=\SI{130}{\volt}$, $\alpha = \SI{20}{\degree}, $ $B_p=\SI{90}{\volt}$, $\beta = \SI{-40}{\degree}$. Obtenga las gráficas de las siguientes ondas en el rango desde t=\SI{0}{\second} hasta t=\SI{1/40}{\second}  usando Microsoft \textregistered Excel:
\begin{itemize}
    \item[] $a = A_p \cdot \sin(\omega t + \alpha )$
    \item[] $b = B_p \cdot \sin(\omega t + \beta )$
\end{itemize}
\noindent\textbf{Aclaraciones:}
\begin{itemize}
    \item Recuerde utilizar una división temporal que permita visualizar la onda correctamente.
    \item Ambas gráficas deben entregarse en un solo archivo de Excel con la descripción y las unidades adecuadas en los ejes.
    \item El archivo final debe nombrarse de la siguiente manera:\\ \textbf{PrimerApellido\_PrimerNombre\_Tarea1.xlsx}
    \item El archivo debe enviarse al correo electrónico \textbf{juan.rojas@ieee.org} el día indicado antes de la media noche
    \item Cualquier entrega tardía se califica por debajo de 70. 
\end{itemize}
\end{document}