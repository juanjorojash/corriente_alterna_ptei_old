\documentclass[12pt]{article}
\usepackage[utf8]{inputenc}
\usepackage[spanish,activeacute]{babel}
\usepackage[margin=1in]{geometry}
\usepackage{amsmath,amssymb}
\usepackage{multicol}
\usepackage{siunitx}
\usepackage[american]{circuitikz} %Este se usa para hacer circuitos
\usepackage{float}
\usepackage{hyperref}
\usetikzlibrary{babel}




% *** GRAPHICS RELATED PACKAGES ***
%
\usepackage{graphicx}
\graphicspath{{../pdf/}{../png/}}
\DeclareGraphicsExtensions{.pdf,.jpg,.png}
\usepackage{subfigure}



% Cambio de nombre de cuadro a Tabla
\renewcommand{\listtablename}{Índice de tablas}
\renewcommand{\tablename}{Tabla}

\begin{document}

\noindent
\begin{tabular*}{\textwidth}{l @{\extracolsep{\fill}} r@{\extracolsep{6pt}}}
\TextField[width=7cm]{\textbf{Nombre:}} &\\	
& \textbf{Informe de Laboratorio}\\
\textbf{Técnico en Electricidad Industrial}& \textbf{Corriente Alterna}\\
\textbf{Fundatec}& \textbf{Realizado: 07/12/2019} \\
\textbf{Tecnológico de Costa Rica}& \textbf{Entrega: 16/12/2019} \\
\end{tabular*}\\

\noindent\rule[2ex]{\textwidth}{2pt}
\noindent\textbf{Instrucciones}
\begin{itemize} \itemsep1pt \parskip0pt \parsep0pt
\item Rellene los espacios indicados de acuerdo a los datos obtenidos durante la experiencia de laboratorio. 
\item Luego de rellenar los campos, el archivo debe nombrarse de la siguiente manera:\\ \textbf{PrimerApellido\_PrimerNombre\_Informe.pdf}
\item El archivo debe enviarse al correo electrónico \textbf{juan.rojas@ieee.org} el día indicado antes de la media noche
\item Cualquier entrega tardía se califica por debajo de 70. 
\end{itemize}

\noindent\rule[2ex]{\textwidth}{2pt}
{\large \textbf{Circuito RC}} \\

\noindent\textbf{Procedimiento}
\begin{enumerate}\itemsep1pt \parskip0pt \parsep0pt
\item Arme el circuito de la Figura \ref{fig:F1} y conecte el osciloscopio digital tal y como se muestra. 
\item Obtenga una onda senoidal usando el generador de señales. 
\item Utilice los cursores del osciloscopio para encontrar el desfase entre $V_R$ y $V_C$.
\end{enumerate}

\end{document}
